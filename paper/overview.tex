\section{dBoost overview}
\label{sec:overview}

We designed a framework that analyzes, models, and detects outliers in data.
The whole system can be seen in Figure~\ref{fig:pipeline}.

As a first step, tuples are read from the database and expanded: a set of type-dependent features is extracted for each field. These features express simple properties of the data, such as the length of a string, or the parity of an integer.

These expanded tuples are then analyzed in order to obtain simple statistical information, and to detect soft functional dependencies between different fields. The expanded tuples are then used to train one of three data models (Gaussian, Mixtures, or Histograms), with the help of the statistics and correlation hints gathered at the previous stage.

Finally, the trained model is used to classify tuples into regular records and outliers; these tuples can be the ones the model was trained with, or future inputs to the database system.

From a high level view, our pipeline is implemented as a three-pass streaming algorithm, requiring no memory beyond that required to train the individual models.

The different components of our system are summarized as follows and described in detail in the following sections:

\begin{enumerate}
\item Preprocessing -- Tuples are expanded using knowledge about the database schema and field types (Section~\ref{sec:preprocessing}).
\item Statistical analysis -- The expanded data is analysed to gather basic statistics, along with correlation information. These statistics are used for modeling and outlier detection (Section~\ref{sec:statistical-analysis}).
\item Data modeling -- We apply various machine-learning algorithms to build models of the data (Section~\ref{sec:model-creation}).
\item Outlier detection -- Using the models built in the previous stage and user-provided sensitivity thresholds, we report outliers identified by the models trained during the previous stage (Section~\ref{sec:outlier-detection}).
\end{enumerate}

In Table~\ref{tab:example} we show an example of our tool applied to a small dataset that includes a transaction ID, a registration date (Reg. Date), and a social security number (SSN).
Our tool reads the data row-by-row and expands the registration date based on the integer field type into additional columns, of which two, year and weekday, are shown.
Similarly, SSN is also expanded based on its string field type to the string length and a copy of the string with numbers stripped out and replaced by $<$num$>$.
These tuple expansions are not materialized in the database, but rather fed into the model one-by-one, after which the expansions for each row are discarded.
The outliers detected by the Histogram model are highlighted in red in the table. 

\begin{table*}[t]
\begin{center}
\begin{tabular}{|c|c|c||c|c|c||c|c|c|}
\multicolumn{3}{c}{Original Data} & \multicolumn{3}{c}{Tuple Expansions of Int} & \multicolumn{3}{c}{Tuple Expansions of SSN} \\
\multicolumn{3}{c}{} & \multicolumn{3}{c}{(not materialized)} & \multicolumn{3}{c}{(not materialized)} \\
\hline
XID & Reg. Date & SSN & Year & Weekday & ... & Length & Strip Numbers & ... \\ \hline
1 & 1416497422 & 783-345-2351 & 2014 & Thursday && 12 & $<$num$>$-$<$num$>$-$<$num$>$ &...  \\ \hline 
\rowcolor{red} 
2 & 1418201134 & 773-746- & 2014  & Wednesday &...& 8 & $<$num$>$-$<$num$>$- &...  \\ \hline 
3 & 1420359855 & 773-289-5552 & 2015  & Sunday &...& 12 & $<$num$>$-$<$num$>$-$<$num$>$ &...  \\ \hline 
4 & 1421575392 & 849-843-2729 & 2015 & Sunday &...& 12 & $<$num$>$-$<$num$>$-$<$num$>$ &  ...\\ \hline 
\rowcolor{red} 
5 & 01302015 & 773-387-9201 & 1970 & Friday &...& 12 & $<$num$>$-$<$num$>$-$<$num$>$ &  ...\\ \hline 
6 & 1424866716 & 821-322-1857 & 2015 & Wednesday &...& 12 & $<$num$>$-$<$num$>$-$<$num$>$ &  ...\\ \hline 
7 & 1425059692 & 822-971-1892 & 2015 & Friday &...& 12 & $<$num$>$-$<$num$>$-$<$num$>$ &  ...\\ \hline 
%1 & 1416497422 & 11/20/2014 & 15:30:32 & Thursday & 2 \\ \hline 
%2 & 1418201134 & 12/10/2014 & 8:45:34 & Wednesday & 4 \\ \hline 
%3 & 1420359855 & 1/4/2015 & 8:24:15 & Sunday & 5 \\ \hline 
%4 & 1421575392 & 1/18/2015 & 10:03:12 & Sunday & 2 \\ \hline 
%\rowcolor{red} 5 & 01302015 & 1/16/1970 & 1:40:15 & Friday & 5 \\ \hline 
%6 & 1424866716 & 2/25/2015 & 12:18:36 & Wednesday & 6 \\ \hline 
%7 & 1425059692 & 2/27/2015 & 17:54:52 & Friday & 2 \\ \hline 
\end{tabular}
\caption{An example dataset showing outliers detected by a histogram model.}
\label{tab:example}
\end{center}
\hspace{6.5cm} 
\includegraphics{../graphics/table-histograms-year}
\hspace{2.8cm} 
\includegraphics{../graphics/table-histograms-length}
\hspace{1.5cm} 
\includegraphics{../graphics/table-histograms-sig}

\end{table*}