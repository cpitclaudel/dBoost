% THIS IS AN EXAMPLE DOCUMENT FOR VLDB 2012
% based on ACM SIGPROC-SP.TEX VERSION 2.7
% Modified by  Gerald Weber <gerald@cs.auckland.ac.nz>
% Removed the requirement to include *bbl file in here. (AhmetSacan, Sep2012)
% Fixed the equation on page 3 to prevent line overflow. (AhmetSacan, Sep2012)

\documentclass{vldb}
\usepackage{graphicx}
\usepackage{balance}  % for  \balance command ON LAST PAGE  (only there!)
\usepackage{url}
\usepackage{hyperref}
\usepackage{microtype}
\usepackage{calc}
\usepackage[utf8]{inputenc}

\hypersetup{breaklinks=true,
            bookmarks=true,
            pdfauthor={},
            pdftitle={},
            colorlinks=true,
            citecolor=blue,
            urlcolor=blue,
            linkcolor=magenta,
            pdfborder={0 0 0}}

\DeclareMathOperator{\Var}{Var}
\DeclareMathOperator{\Covar}{Covar}
\begin{document}

% ****************** TITLE ****************************************

\title{User guided outlier detection in heterogeneous datasets}

% possible, but not really needed or used for PVLDB:
%\subtitle{[Extended Abstract]
%\titlenote{A full version of this paper is available as\textit{Author's Guide to Preparing ACM SIG Proceedings Using \LaTeX$2_\epsilon$\ and BibTeX} at \texttt{www.acm.org/eaddress.htm}}}

% ****************** AUTHORS **************************************

% You need the command \numberofauthors to handle the 'placement
% and alignment' of the authors beneath the title.
%
% For aesthetic reasons, we recommend 'three authors at a time'
% i.e. three 'name/affiliation blocks' be placed beneath the title.
%
% NOTE: You are NOT restricted in how many 'rows' of
% "name/affiliations" may appear. We just ask that you restrict
% the number of 'columns' to three.
%
% Because of the available 'opening page real-estate'
% we ask you to refrain from putting more than six authors
% (two rows with three columns) beneath the article title.
% More than six makes the first-page appear very cluttered indeed.
%
% Use the \alignauthor commands to handle the names
% and affiliations for an 'aesthetic maximum' of six authors.
% Add names, affiliations, addresses for
% the seventh etc. author(s) as the argument for the
% \additionalauthors command.
% These 'additional authors' will be output/set for you
% without further effort on your part as the last section in
% the body of your article BEFORE References or any Appendices.

\numberofauthors{3} 

\newcommand{\MITemail}[1]{\email{#1@mit.edu}}
\author{
% You can go ahead and credit any number of authors here,
% e.g. one 'row of three' or two rows (consisting of one row of three
% and a second row of one, two or three).
%
% The command \alignauthor (no curly braces needed) should
% precede each author name, affiliation/snail-mail address and
% e-mail address. Additionally, tag each line of
% affiliation/address with \affaddr, and tag the
% e-mail address with \email.
%
% 1st. author
\alignauthor
Clément Pit-\kern0pt-Claudel\\
       \affaddr{MIT CSAIL}\\
       \MITemail{cpitcla}
% 2nd. author
\alignauthor
Zelda Mariet\\
       \affaddr{MIT CSAIL}\\
       \MITemail{zmariet}
% 3rd. author
\alignauthor
Rachael Harding\\
       \affaddr{MIT CSAIL}\\
       \MITemail{rhardin}
%\and  % use '\and' if you need 'another row' of author names
}
\date{10 December 2014}

\maketitle

\begin{abstract}
Rapidly developing areas of information technology are generating massive amounts of data. Human errors, sensor failures, and other unforeseen circumstances unfortunately tend to undermine the quality and the consistency of these datasets by causing the apparition of outliers -- data points that exhibit surprising behavior when compared to the rest of the data. Characterizing, locating, and in some cases eliminating these outliers offers interesting insight about the data under scrutiny and reinforces the confidence that one may have in conclusions drawn from otherwise noisy datasets.

We report on the design and development of a new user-guided outlier detection framework, dBoost, which relies on inference and statistical modeling of heterogeneous data to flag suspicious fields in database tuples. At the heart of the system lies a tuple expansion procedure, which reconstructs rich information from semantically poor SQL data types such as strings, integers, and floating point numbers. We show that this novel approach achieves good classification performance, both in traditional numerical datasets and in highly non-numerical contexts such as mostly textual datasets. Our implementation is publicly available, under version 3 of the GNU General Public License.
\end{abstract}

\section{Introduction}
\label{sec:intro}
% INTRODUCTION

%Detecting outliers is an important problem in detecting inconsistencies within a data set.
Sensor glitches, suspicious activity, and human input error can all result in the insertion of outlier rows in a database. If these rows go undetected, their presence can be detrimental to later database operations.

Formally, an outlier is defined as “an observation which deviates so much from the other observations as to arouse suspicions that it was generated by a different mechanism” \cite{Hawkins1980}.

While previous work has focused on detecting outliers before performing data analysis, no tools to our knowledge have been built to not only automatically detect, but also explain why outliers fall outside the expected data behavior.

In this work, we draw on some of the ideas from existing literature on statistical and probabilistic methods for outlier detection. We combine these into a framework that makes three passes over the data in order to analyze, model, and discover outliers in the data.

Our framework can be expanded naturally so that analysis and modeling can be done offline on a small sampling of the dataset. The data models can then be used to detect outliers in new data that is added to the database. Hence, outliers can be detected on-line without re-running any analysis on the rest of the data.

Our contributions are as follows:
\begin{enumerate}
\item We present a framework that detects outliers, using tuple expansion to extract relevant information from the data.
\item We apply machine learning techniques to model data behavior.
\item We evaluate the performance of our framework on several real-world datasets.
\end{enumerate}

We built a tool that utilizes our framework; it is publicly available under the GNU Public License~\cite{github}.

The rest of this paper is structured as follows:
We first provide an overview of our framework in Section~\ref{sec:overview}.
In Section~\ref{sec:implementation}, we delve into the details of how we built a software tool to implement our framework.
We evaluate our tool on several real-world problems in Section~\ref{sec:eval}.
We describe related work in Section~\ref{sec:related_work}, followed by our plans for future work in Section~\ref{sec:future} and conclusions from the project.


\section{Framework Overview}
\label{sec:overview}
\section{dBoost overview}
\label{sec:overview}

We designed a framework that analyzes, models, and detects outliers in data.
The whole system can be seen in Figure~\ref{fig:pipeline}.

\begin{figure*}
  \centering %TODO: Should this be full width?
  \paddedgraphics[width=.8\textwidth]{../graphics/pipeline.pdf}
  \caption{The dBoost pipeline}
  \label{fig:pipeline}
\end{figure*}

As a first step, tuples are read from the database and expanded: a set of type-dependent features is extracted for each field. These features express simple properties of the data, such as the length of a string, or the parity of an integer.

These expanded tuples are then analyzed in order to obtain simple statistical information, and to detect soft functional dependencies between different fields. The expanded tuples are then used to train one of three data models (Gaussian, Mixtures, and Histograms), with the help of the statistics and correlation hints gathered at the previous stage.

Finally, the trained model is used to classify tuples into regular records and outliers; these tuples can be the ones the model was trained with, or future inputs to the database system.

From a high level view, our pipeline is implemented as a three-pass streaming algorithm, requiring no memory beyond that required to train the individual models.

The different components of our system are summarized as follows and described in detail in the following sections:

\begin{enumerate}
\item Preprocessing -- Tuples are expanded using knowledge about the database schema and field types (Section~\ref{sec:preprocessing}).
\item Statistical analysis -- The expanded data is analysed to gather basic statistics, along with correlation information. These statistics are used for modeling and outlier detection (Section~\ref{sec:statistical-analysis}).
\item Data modeling -- We apply various machine-learning algorithms to build models of the data (Section~\ref{sec:model-creation}).
\item Outlier detection -- Using the models built in the previous stage and user-provided sensitivity thresholds, we report outliers identified by the models trained during the previous stage (Section~\ref{sec:outlier-detection}).
\end{enumerate}

\section{Framework Implementation}
\label{sec:implementation}
\subsection{Preprocessor}
\label{sec:preproc}
% PREPROCESSOR

We preprocess the data by first discarding tuples that do not have the correct number of tuples.
We then use a tuple expander to gather more information from the data, which we describe in the next section.

\subsubsection{Tuple Expansion}

\noindent\begin{minipage}{0.8\linewidth}
  \itshape
  What if you need to store a date and time value with subsecond resolution? MySQL
  currently does not have an appropriate data type for this, but you can use your own
  storage format: you can use the BIGINT data type and store the value as a timestamp in
  microseconds, or you can use a DOUBLE and store the fractional part of the second after
  the decimal point.
\end{minipage}
\begin{flushright}
  \textit{High-Performance MySQL}, 3\textsuperscript{rd} edition (2012), p127
\end{flushright}

Often data stored in databases has meaning, such as dates or times.
However, the semantics of SQL are not expressive enough to use them in outlier analysis.
For example, the day of the week may be relevant information from a date in a banking application in which transactions are only completed on weekdays, but unless the database stores this information in a separate column, the information is lost.
We expand tuples in order to harness isolated information about the data that may not be easy to see when looking at the data in its original form.

In order to capture this extra information, we break down tuples into a series of sub-tuples based on the type of each column along with the original value, as shown in Figure~\ref{TODO}.
We determined these breakdowns based on our own intuition and the datasets we worked with.

Our tool currently expands three kinds of data types: strings, dates, and integers.
All other types are passed to the other stages in the framework without modification.
Our tool automatically determines the data type as tuples are read in, although the tool could be modified to get this information from table schemas.

The expansions we do on strings includes indicating case, determining whether there are digits in the string, and the string length.
Dates are broken down into year, month, day, hour, minute, seconds, and day of the week.
Integers are expanded to isolate individual bit activation and modularity.

\begin{figure}
  \newenvironment{stackedlines}{\renewcommand{\arraystretch}{1.2}\begin{array}[b]{@{}l@{\quad}l@{}}}{\end{array}}
  $\begin{stackedlines}
    \texttt{string:}\\
    \texttt{\parbox{\widthof{1418222134.325}}{"32-G414"}}
  \end{stackedlines} \longrightarrow
  \begin{cases}
    \text{length: } & \texttt{7}\\
    \text{pattern: } & \texttt{NNPLNNN}\\
    \text{uppercase: } & \texttt{True (1)}\\
    \text{lowercase: } & \texttt{False (0)}\\
    \text{\parbox{\widthof{base-10 residue:}}{title case:} } & \texttt{True (1)}
  \end{cases}$

  $\begin{stackedlines}
    \texttt{int:}\\
    \texttt{\parbox{\widthof{1418222134.325}}{1418222134}}
  \end{stackedlines} \longrightarrow
  \begin{cases}
    \text{date: } & \texttt{(2014,12,10)}\\
    \text{time: } & \texttt{(14,35)}\\
    \text{weekday: } & \texttt{Wed (2)}\\
    \text{day of year: } & \texttt{344}\\
    \text{binary: } & \texttt{0b10101\ldots010}\\
    \text{base-10 residue: } & \texttt{4}
  \end{cases}$

   $\begin{stackedlines}
    \texttt{float:}\\
    \texttt{1418222134.325}
  \end{stackedlines} \longrightarrow
  \begin{cases}
    \text{intpart: } & \texttt{1418222134}\\
    \text{decpart: } & \texttt{0.325}\\
    \text{millis: } & \texttt{325}\\
    \text{\parbox{\widthof{base-10 residue:}}{date, \ldots:} } & \texttt{\ldots}
  \end{cases}$

  \caption{Summary of tuple expansion rules currently implemented.}
  \label{fig:tuple-expansion}
\end{figure}

\subsection{Statistical Analysis}
\label{sec:stat_anal}
% STATISTICAL ANALYSIS
Following the tuple expansion phase, the new expanded tuples go through a statistical analysis phase. This phase collects two kinds of information: simple statistics on each column's values, and data on possible inter-column correlations.

The statistical information includes the average value of each column, its variance and its standard deviation. These statistics can then be used by the outlier models to improve their performance.

Furthermore, if there is more than one column in the original dataset, we run an additional analysis to determine whether there are correlations between columns in the expanded sub-tuples.

This analysis is done using the Pearson product-moment correlation. This simple statistical method is commonly used to find correlation between multiple values. It relies on the Pearson product-moment correlation coefficient, which measures linear correlation between two vectors.

Given two column vectors $X$ and $Y$, Pearson's coefficient $R$ is given by the following formula:

\begin{equation} 
\label{eqn:pearson}
R = \frac{\Covar(X,Y)}{\sqrt{\Var(X)\Var(Y)}}
\end{equation}

$R$'s value always lies between -1 and 1. An $R$ value close to 0 indicates little or no correlation. The closer the coefficient is to 1, the more closely correlated the columns are; the closer it is to -1, the more inversely correlated the columns are.

If a sufficient correlation or inverse correlation is found, the statistical analyzer sends information (``hints'') about the tuples and sub-tuples that are correlated to the outlier models.

Currently, we only detect correlations between expanded fields of different columns, as they are more likely to be meaningful. However, although some correlations in expanded fields of the same column are redundant (\texttt{isUppercase} and \texttt{isLowercase} for a string), some might be relevant (for example, a correlation between \texttt{weekday} and \texttt{month} for a date). We leave this exploration to future work.

In our implementation, all statistics and correlations are computed in memory using a single pass over the data; the expanded tuples are analyzed one row at a time, and the final statistics and correlation coefficients are computed after the last tuple has been processed.

The analyzer's results are accessible by any models used at later stages in the tool.

\subsection{Data modeling}
\label{sec:model_creation}
% MODEL CREATION

\subsubsection{Histogram Model}
\subsubsection{Gaussian Model}
The Gaussian model assumes that each expanded column is independent from the others, and that all values in a column $c$ are independent values drawn from a normal distribution $\mathcal N(\mu_c, \sigma_c)$.

The model's parameters (a pair $(\mu, \sigma)$ for each numerical column) are obtained with one pass over the data, without loading the tuples in memory.

\subsubsection{Mixture Model}

The Mixture Model (MM) assigns a Gaussian Mixture Model (GMM) to each correlation between numerical valued-columns that has been detected during the statistical analysis. These GMMs model the probability distribution of the correlated values.

For example, if the preprocessor detects that field 2 of expanded column 1 and field 1 of expanded column 3 are correlated, the model will learn a GMM to model this correlation: the probability of obtaining $(X_1, X_2)$ as values for the two correlated fields is given by
\[\Pr(X_1, X_2) = \sum_{j=1}^{n} \pi_j \mathcal N(\mu_j, \Sigma_j)(X_1, X_2)\]
where $n$ is the number of components chosen for the GMM (we set $n=2$, as learning this parameter would severely increase the time necessary for learning the mixture model), and $\pi_j, \mu_j$ and $\Sigma_j$ are parameters of the GMM. 

The Mixture Model learns all  $\pi_j, \mu_j$ and $\Sigma_j$ parameters for each correlation detected during the preprocessing phase. This was implemented using python's \texttt{scikit-learn} library; the process loads all correlations to memory. We expect that on average, the set of correlations will be much smaller than the set of rows themselves, thus limiting memory usage.


\subsection{Outlier Detection}
\label{sec:outlier_detection}
% OUTLIER DETECTION

\subsubsection{Histogram Model}
\subsubsection{Gaussian Model}
\subsubsection{Gaussian Mixture Model}


\section{Evaluation}
\label{sec:eval}
% Evaluation

We evaluated our framework by implementing a tool in python3.
We have approximately TODO lines of code.
Our code is publicly available on GitHub under a GNU Public License~\cite{github}.
The program loads data from a text file (column separators can be specified on the command line, so many formats are possible), does computation, and reports outliers on the command line.
Table~\ref{table:flags} shows some of the usage of our code.

\begin{table*}
\label{table:flags}
\caption{Tool usage.}
\centering
\begin{tabular} {| l | l | p{10cm} |}
\hline
\multicolumn{3}{|c|}{./dboost-stdin.py input-file} \\
\hline
Flag & Options & Explanation \\
\hline
--gaussian & n\_stdev & Report outliers that fall more than n\_stdev standard deviations away from the mean of the data \\
--mixture & n\_subpops & Use a model of n\_subpops gaussians \\
--histogram & peak\_s & Consider only fields with a peaked distribution with peakiness peak\_s \\
  & outlier\_s & Report values that fall in classes with less than outlier\_s percent \\
--statistical & epsilon & Give hints to the model for correlations with Pearson R coefficient greater than epsilon \\
\hline
\end{tabular}
\end{table*}

We ran our tool on several real-world data sets with varying schemas and properties.
We analyze each data set separately in the following sections.

\subsection{CSAIL Directory}
\label{sec:csail}
% CSAIL Directory
The CSAIL directory is an online directory of faculty, staff and students in the MIT Computer Science and Artificial Intelligence Laboratory \cite{CSAILDirectory}.
Each entry contains a person's name, phone number, office number, email address, and position.
Some data, such as a phone number, may be missing from the directory.

%Histograms


\subsection{Intel Lab Data}
\label{sec:intel}
% INTEL LAB DATA

We evaluated our outlier detection framework on sensor data from the publicly available Intel Lab Data set \cite{IntelLabData}.
The Intel Lab Data contains data collected from 54 sensors spread throughout the Intel Berkeley Research Lab.
Each data entry is timestamped and contains information including humidity, temperature, light and voltage taken from a Micro2dot sensor and weatherboard.
The dataset includes approximately 2.3 million entries, but in our experiments we drew from a random sample of the rows.

This data set has known outliers from faulty sensor readings due to periods of critically low voltage.
Most notably, sensor 15 fails and 


%\subsection{Presidential Campaign Finance}
%\label{sec:president}
%% PRESIDENTIAL CAMPAIGN DATA
\cite{PresCampaignData}


%\subsection{Mimic2}
%\label{sec:mimic2}
%% MIMIC2 DATA ANALYSIS


\section{Related Work}
\label{sec:related_work}
% RELATED WORK

% Outlier Detection
There has been substantial research in how to build models to detect outliers \cite{Aggarwal2013}, including how to detect outliers in high­dimensional data by searching the subspaces of the dat \cite{Zhang2004}\cite{Kriegel2009}.
However many of these algorithms are complex and can require substantial computation to determine whether a new data point lies outside the data.

% Outlier Detection Visualization
Additional research has been done to try to summarize outliers in data visualization techniques \cite{Wu}.

% Statistical methods
Statistical methods have been used to detect functional dependencies between columns of relational databases for the purpose of informing the query optimizer of potential data dependencies \cite{Ilyas2004} 
These methods require only a small sample of the data to detect functional dependencies with high probability of correctness.
The relatively low computation required by these algorithms makes them more amenable to detecting data anomalies in real time.

% Gaussian models




\section{Future Work}
\label{sec:future}
% FUTURE WORK

In our future work, we would like to expand our framework to include more statistical and probabilistic models.



\section{Conclusion}
\label{sec:concl}
% CONCLUSION

In this paper, we presented dBoost, a toolkit that detects outliers in both real-valued and heterogeneous data sets using three different data modeling techniques.

We demonstrated how the tool performs on real-world problems, including identifying potential wrong entries in a directory and flagging erroneous values generated by faulty sensors.
This toolkit is available for public use to allow developers to formulate their own type-based rules and find discrepancies in their own data.

We believe that the preliminary results presented in this paper are promising, especially in the area of identifying outliers in heterogeneous data.



\section{Acknowledgments}
The authors would like to thank Eugene Wu (MIT CSAIL) for providing guidance on dataset selection, and Samuel Madden (MIT CSAIL) for guidance and advice in the course of this project.

% The following two commands are all you need in the
% initial runs of your .tex file to
% produce the bibliography for the citations in your paper.
\bibliographystyle{abbrv}
\bibliography{paper.bib}  % vldb_sample.bib is the name of the Bibliography in this case
% You must have a proper ".bib" file
%  and remember to run:
% latex bibtex latex latex
% to resolve all references

%\subsection{References}

%APPENDIX is optional.
% ****************** APPENDIX **************************************
% Example of an appendix; typically would start on a new page
%pagebreak

%\begin{appendix}

%\end{appendix}



\end{document}

